\resumo
\addcontentsline{toc}{chapter}{Resumo}

Devido ao curto tempo de reação a estímulos sonoros do aparelho auditivo humano, para performar em conjunto com outros artistas, músicos necessitam que haja pouca latência entre a saída de dos instrumentos de seus colegas e seu retorno. Dessa forma, proporcionar um ambiente colaborativo em tempo real via Internet torna-se um desafio pertinente na área de Computação Musical e Rede de Computadores. Algumas soluções procuram otimizar a conexão entre os computadores construindo infraestruturas dedicadas ou abandonam o requisito de tempo real ao entregar experiências assíncronas. No entanto, tais abordagens não abrangem, de forma síncrona, casos em que não haja acesso a uma conexão de alta velocidade ou que exista uma grande distância entre os participantes.

Este trabalho, portanto, investiga a possiblidade de predizer as próximas sequências imediatas de áudio, baseando-se em entradas passadas, considerando dois métodos - predição de sequências com LSTM (\textit{Long Short-Term Memory})\cite{lstm} e indexação de sequências usando DTW (\textit{Dynamic Time Warping})\cite{dtw}. De tal forma, espera-se que, não sendo necessária a espera pela saída do cliente transmissor, haja uma redução da percepção de latência por parte dos participantes.

\begin{keywords}
latência, áudio, predição de sequência, streaming, rollback, dtw, lstm
\end{keywords}
