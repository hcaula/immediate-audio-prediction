\section{Estado da arte}

% quais são as soluções atuais para o problema? Como elas atendem os "critérios de avaliaçao"? O que pode serve de inspiração para sua solução?
% no caso de problema/solução original, faz um estado da arte de delimitação do escopo (eu conheo o que existe o suficiente para dizer que ninguém)

Como forma de reduzir o efeito da latência em transmissões \textit{online} de música, nós propomos a previsão local de áudio baseado em entradas anteriores. Não foram encontradas soluções, estudos ou aplicações que procuram resolver esse problema dessa forma.

No entanto, para resolver o problema da latência em ambientes musicais, identificamos duas abordagens principais: (1) otimizações na camada de transporte da Internet e (2) criação de ambientes assíncronos, onde os músicos não performam em tempo real uns com os outros.

\subsection{Otimizações na camada de transporte}
\label{sec:delay-based-audio-solutions}

Esta abordagem procura atacar o problema de forma mais linear, implementando melhorias na conexão pela Internet entre os participantes ou utilizando infraestruturas de rede específicas.

\textit{LoLa} (\textit{LOw LAtency audio visual streaming system}) \cite{lola}, um sistema de \textit{streaming} audio visual desenvolvido pelo \textit{Conservatorio di Musica G. Tartini}, consegue atingir conexões de baixa latência, utilizando infraestruturas de rede dedicadas. Foi usado em diversas apresentações de até 3.500 km de distância entre os músicos entre os anos de 2010 e 2013 \cite{lola_streaming}. No entanto, deixa muito claro que sua solução não é necessariamente acessível, sendo recomendado no mínimo 1 Gigabit por segundo de banda em todas as pontas. De acordo com estudo realizado pela Viavi Solutions em 2019, apenas 5\% da população mundial possui conexões tão rápidas \cite{1gbps}, e a média de velocidade mundial é de apenas 97.52 Megabits por segundo, pelos dados apresentados pelo Speedtest Global Index \cite{speed_test}. Portanto, para o público em geral, não é uma solução de fácil acesso.

\textit{SoundJack} \cite{soundjack}, por outro lado, utiliza-se de alguns métodos que o torna mais acessível para o usuário comum. Ele consegue atingir velocidades mais rápidas que aplicações comerciais como \textit{Zoom}, \textit{Teams} e \textit{FaceTime} por dois fatores: (1) conecta usuários diretamente através de P2P (\textit{peer-to-peer}), ao contrário das soluções citadas, que transportam dados entre servidores centrais e (2) não otimiza a qualidade o áudio/vídeo automaticamente, deixando como responsabilidade do usuário; caso prefiram, os músicos podem optar por conexões de menores latências assumindo o custo de obter piores qualidades de áudio. Nas configurações recomendadas, infraestruturas comerciais comuns residenciais são suportadas via Ethernet. Entretanto, \textit{SoundJack} requer um poder computacional relativamente alto para atingir baixas latências - recomenda no mínimo processadores i7 Quad Core, custando cerca de R\$2456,90 \footnote{Preço encontrado na Amazon Brasil em 04/04/2021.}, também oferecendo um dispositivo próprio dedicado à aplicação, vendendo por €226,05 \footnote{Preço encontrado na Symonics em 04/04/2021.}.

A aplicação comercial \textit{JamKazam} \cite{jamkazam} também baseia-se em entrega síncrona dos pacotes de áudio para construir um ambiente musical colaborativo \textit{online}. De acordo com seus os resultados apresentados, é possível performar em conjunto com baixas latências onde os músicos estejam em um mesmo estado, numa distância média de 490 km \cite{jamkazam_video}. Porém, entre maiores distâncias ou infraestruturas não baseadas em fibras, a latência excelente recomendada pela aplicação de 30 ms \cite{jamkazam_latencies} é difícil de ser obtida.

Em termos de acessibilidade para o público geral, a solução \textit{open source} \textit{JackTrip} fornece guias de instalação para Raspbery Pi \cite{jacktrip_rasp}, além de uma instalação simples para Linux, OS X e Windows. Também de código aberto, o SonoBus \cite{sonobus} fornece um \textit{download} gratuito, além de guias de instalação e dicas para melhoria na latência.

\subsection{Criação de ambientes assíncronos}

Uma solução que possibilita a percepção de baixa latência é abdicar do requisito de entregar uma experiência em tempo real. Caso seja possível atrasar a entrega dos pacotes de forma não perceptível e, portanto, aumentando a janela mínima de espera, não é necessário possuir baixas latências.

A aplicação \textit{jammr} \cite{jammr} aproveita o conceito de progressão de acordes da teoria musical a seu favor. A cada \textit{loop}, os participantes ouvem o que foi tocado na última iteração pelos seus colegas. Dessa forma, por não ser necessário manter as performances em sincronia, há uma tolerância muito maior às latências causadas tanto pela camada de transporte como as causadas pelos processamentos de áudio locais. No entanto, tal solução necessita que os músicos toquem a mesma progressão em \textit{loop}, funcionando bem para improvisações simples (\textit{jam sessions}); porém, para músicas mais complexas, o sistema não é ideal.

O Jamtaba \cite{jamtaba}, um cliente \textit{open source} de servidores NINJAM \cite{ninjam}, utiliza uma solução semelhante de assincronia. A aplicação funciona utilizando latências grandes, porém sincronizadas entre os participantes. Para cada intervalo, os músicos gravam o áudio e, uma vez finalizado, a gravação é reproduzida pelos outros clientes. Então, todos ouvem a última gravação de seus colegas, porém, ouvindo a si mesmo em tempo real.

\subsection{Considerações}

Cada abordagem para o problema da latência apresenta um conjunto de vantagens e desvantagens entre si. Dessa forma, podemos compará-las na \tabref{tab:morfological_matrix} de acordo com quatro qualidades: (1) se entrega os áudios em tempo real; (2) se oferece baixo custo financeiro para usá-la; (3) se funciona bem em conexões lentas, de acordo com as valores demonstrados na \secref{sec:problem} e; (4) se suporta improvisações dos músicos. 

\renewcommand{\arraystretch}{2}

\begin{table}[ht!]
    \centering
    \begin{tabular}{|c|c|c|c|c|c|}
        \cline{3-6}
        \multicolumn{2}{c|}{} & \rotatebox[origin=c]{90}{\makecell{Tempo \\ real}} & \rotatebox[origin=c]{90}{\makecell{Baixo \\ custo}} & \rotatebox[origin=c]{90}{\makecell{Funciona em \\ conexões lentas}} & \rotatebox[origin=c]{90}{\makecell{Suporta \\ improvisações}} \\
        \hline
        
        \multirow{5}{5em}{\centering Soluções síncronas} & LoLa & X & & & X \\ 
        \cline{2-6}
        & SoundJack & X & & & X  \\ 
        \cline{2-6}
        & JamKazam & X & & & X \\ 
        \cline{2-6}
        & JackTrip & X & X & & X \\ 
        \cline{2-6}
        & SonoBus & X & X & & X \\ 
        \cline{2-6}
        
        \hline
        \hline
        
        \multirow{2}{5em}{\centering Soluções assíncronas} & jammr & & X & X & X \\ 
        \cline{2-6}
        & JamTaba & & X & X & X \\ 
        \cline{2-6}
        
        \hline
        \hline
        
        \multirow{1}{5em}{\centering Solução proposta} & Client-side prediction & X & X & X & \\ 
        \hline
    \end{tabular}
    \caption{Matriz morfológica comparando as diferentes abordagens para solucionar o problema da latência em ambientes musicais colaborativos \textit{online}.}
    \label{tab:morfological_matrix}
\end{table}

Observa-se, portanto, que nenhuma das soluções avaliadas consegue fornecer, simultaneamente, uma experiência em tempo real, que seja acessível e funcione em conexões lentas. Nossa solução, portanto, propõe na \secref{sec:client_side_prediction} uma adaptação do algoritmo de \textit{client-side prediction}, utilizada em videogames \textit{online}, que proporciona uma experiência com essas três características.

Pela natureza preditiva desse algoritmo, improvisações musicais não podem ser suportadas, umas vez que sua tendência imprevisível torna essa tarefa bastante desafiadora. Outras soluções, que reproduzem o áudio produzido pelos músicos fidedignamente, cumprem esse requisito perfeitamente.
