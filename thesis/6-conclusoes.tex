\chapter{Conclusões}
\label{chap:conclusion}

De acordo com nosso experimentos, o modelo preditivo gerador de novas sequências com LSTM não teve bom desempenho em nenhuma das duas métricas de sucesso estabelecidas na \secref{sec:success_metrics}. Por repetir os dados de entrada, caso utilizássemos esse modelo em uma aplicação real, estaríamos apenas atrasando o áudio transmitido e replicando o problemas que as soluções síncronas enfrentam hoje.

Além disso, devido ao alto tempo necessário para realizar a predição, tal solução mostrou-se inviável da forma que foi implementada. Para atingir tempos menores, seria necessário mais poder computacional, tornando essa solução inacessível para a população em geral.

Portanto, de acordo com nossos experimentos, o modelo gerador com LSTM não se mostrou promissor para ser utilizado em uma adaptação do \textit{client-side prediction} para ambientes musicais. Porém, melhorias podem ser estudadas que viabilizem seu uso, exploradas na \secref{sec:later_work}.

Entretanto, no segundo ciclo, o modelo indexador e identificador mostrou-se bastante promissor em nossos experimentos. Apesar de possuir uma base da dados pequena, a taxa de acerto das previsões foi consideravelmente alta, de forma a trazer a sensação de fluidez nos momentos de acerto na previsão. Se relacionarmos com o \textit{client-side prediction}, que corrige os \textit{inputs} dos jogadores sempre que há um erro, os erros nas previsões realizadas podem ser aceitáveis.

Ademais, o tempo necessário para apontar as predições foi bastante baixo, deixando espaço suficiente para que máquinas menos potentes do que a utilizada nos experimentos possam usufruir dessa abordagem.

Porém, não podemos afirmar com certeza suficiente que o modelo indexador pode ser utilizado em aplicações reais. A base de dados utilizada foi propositalmente reduzida para testar a validade do modelo, possibilitando altas taxas de acertos na identificação e alta eficiência. Além disso, as janelas de áudio escolhidas são consideravelmente altas, o que impede seu uso em improvisações musicais.

Entretanto, o modelo indexador com DTW mostrou-se bastante promissor para uso na adaptação do \textit{client-side prediction} para ambientes musicais, principalmente para músicos de base, que não tendem a improvisar em suas performances.

