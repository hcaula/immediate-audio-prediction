\section{Trabalhos futuros}
\label{sec:later_work}

Para o modelo gerador de novas sequências, trazemos nessa Seção métodos que podem aumentar a corretude das previsões. Já para o modelo indexador, exemplificamos como realizar novos experimentos que melhor simulem um ambiente real colaborativo, de forma possui mais dados para afirmar sua viabilidade. Tais métodos podem ser utilizados em trabalhos futuros, visando melhorar os resultados apresentados.

\subsection{Modelo gerador}

Além do LSTM, outros métodos de previsões de sequência podem ser utilizados em trabalhos futuros. Por exemplo, o \textit{seq2seq}, um \textit{framework} codificador/decodificador de propósito geral para o \textit{Tensorflow} \cite{seq2seq}, é utilizado em aplicações que requerem traduções de uma sequências para outras de diferentes domínios. É bastante utilizado em traduções para diferentes idiomas, legendas automáticas de imagens, sumário de textos, entre outros. Sua aplicação em \textit{Time Series Forecasting}, o \textit{seq2seq} considera dois domínios: passado e futuro. Dessa forma, o conjunto de dados constituiria de pares de sequências - o primeiro elemento representando a sequência de áudio tocado e o segundo o que foi reproduzido logo em seguida.

Na \subsecref{subsec:new_sequence_generator}, mencionamos a possibilidade de uso de ferramentas de continuação de música baseada em um estilo como modelo preditivo. Tal abordagem, como trabalhos futuros, pode ser experimentada de outras maneiras. Pachet propõe o instrumento musical \textit{The Continuator}, que é capaz de aprender com as últimas sequências do músico e reproduzir uma continuação \cite{continuator}. Por ser interativo, as continuações são geradas em tempo real, sendo um candidato em potencial para nossa adaptação, dado o requisito de boa eficiência para geração de predições.

No entanto, ainda utilizando o método proposto com LSTM, é possível melhorar os dados de entrada. Em nossos experimentos, utilizamos os sinais digitais diretamente como entrada para aprendizagem e predição do modelo. Tal abordagem pode não ter sido a ideal, uma vez que é difícil extrair informações desses valores em sua forma básica. Além disso, a dimensionalidade e o volume desses dados é bastante grande, dificultando a aprendizagem.

Recomenda-se, portanto, que haja uma extração de informações antes de usar os dados para aprendizagem nas redes neurais. Por exemplo, extração de \textit{features} como o espectrograma e informações de BPM podem ser utilizadas para concentrar as informações dos dados para aprendizagem.

Por exemplo, o Jukebox \cite{jukebox}, mencionado na \subsecref{subsec:new_sequence_generator}, enfrenta problemas semelhantes de dimensionalidade. Sua implementação utiliza a rica técnica de compressão VQ-VAE \cite{vq-vae} para reduzir a dimensão dos dados de áudio, facilitando seu processamento. Portanto, seu uso é bastante pertinente em uma adaptação do \textit{client-side prediction}.

\subsection{Modelo indexador}

Há duas principais melhorias para esse modelo: (1) aumentar a base de dados para torná-la mais generalista e; (2) reduzir a janela de áudio de identificação.

Em nosso experimentos, como a base de dados de referência foi reduzida, foi viável realizar a separação das janelas manualmente. No entanto, em uma aplicação real, tal tarefa teria que ser automatizada. Portanto, para trabalhos futuros, sugerimos que, para atingir o primeiro objetivo, encontre-se métodos que identifiquem sessões repetidas na música e automaticamente as indexe no conjunto de dados de referência.

Além disso, a medida em que a base de referência cresce, mais tempo o DTW levará para identificar as subsequências, dado que sua complexidade está na ordem de $O(N^2)$. Dessa forma, é importante que a base seja pré-processada, de forma a reduzir a base de dados a cada divisão, como uma espécie de \textit{hash table}. Portanto, para trabalhos futuros, sugere-se pesquisas para encontrar técnicas de realizar esse pré-processamento - por exemplo, pode-se indexar essas divisões de acordo com a nota principal daquela janela de áudio.

Como continuação deste trabalho, esperamos implementar esse conjunto de técnicas, com a espectativa de melhoria nos resultados, de forma a possuir dados conclusivos sobre a validade, ou não, a adaptação proposta do \textit{client-side prediction}.