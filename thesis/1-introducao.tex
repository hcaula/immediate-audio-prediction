\chapter{Introdução}

A performance artística musical, quando praticada em conjunto, requer alto nível de colaboração entre os participantes. Em música, sobretudo gêneros com tendências improvisacionais como \textit{jazz}, \textit{blues} e \textit{rock}, o ato de ouvir e reagir ao som de seus companheiros é tão importante quanto aquele produzido individualmente. Dessa forma, o \textit{feedback} auditivo de baixa latência dos instrumentos tocados é fundamental para que haja uma sensação fluida entre os participantes.

Normalmente, músicos performando em conjunto em um mesmo ambiente físico raramente possuem problemas relacionados à latência. No entanto, em um contexto de distanciamento social, encorajado durante à Pandemia de COVID-19, músicos ao redor do mundo viram-se obrigados a transferirem esse ambiente para um virtual \textit{online}. Além dos \textit{delays} causados pelas conversões de sinais analógicos para digitais e vice-versa e do tempo de escrita no \textit{buffer} em memória \cite{how_low_can_you_go}, a latência apresentada pela transmissão de pacotes pela Internet é também apresenta-se como um dos desafios para sobrepor, onde, a depender de fatores como distância entre os músicos e velocidade da rede, pode ser o maior gargalo do processo de \textit{streaming} de áudio.

Aplicações comuns de videoconferências, como \textit{Zoom}, \textit{Google Meet} e \textit{FaceTime}, que oferecem plataformas para conversações em tempo real, também possuem baixa tolerância a latência, permitindo no máximo 150 ms para manter uma conversa inteligível \cite{cisco} - limite atingível em velocidades medianas de conexões. No entanto, no contexto da prática musical, o limite é ainda menor, variando entre 10 ms e 55 ms \cite{mcphearson}, demonstrando que tais aplicações não podem ser utilizadas para esse propósito. 

Para lidar com esse problema, \textit{softwares} voltados especificamente para a colaboração musical \textit{online} apresentam uma variedade de abordagens diferentes. \textit{LoLa} \cite{lola}, \textit{SoundJack} \cite{soundjack} e \textit{JamKazam} \cite{jamkazam}, por exemplo, implementam otimizações na camada de rede - como conectar clientes diretamente entre si via \textit{P2P} (\textit{peer-to-peer}) - oferecendo latências razoáveis entre distâncias medianas. Outras aplicações, como o \textit{Jammr} \cite{jammr}, dispensam o requisito de tempo real e apresentam soluções assíncronas, nas quais os músicos ouvem os últimos quatro compassos tocados por seus companheiros em um \textit{loop} contínuo.

No entanto, tais abordagens não abrangem casos onde músicos residem entre grandes distâncias ou não é possível ter acesso a conexões dedicadas e \textit{hardwares} de alto valor financeiro, de forma a ainda manter uma performance síncrona.

Ao observar o contexto de videogames, encontramos requisitos de latência similares. Gêneros que utilizam reações como mecânica de jogabilidade, como luta e FPS (\textit{first-person shooter}), para oferecerem aos jogadores uma experiência fluida, necessitam de latências máximas de até 100 ms \cite{pubnub}. O algoritmo mais popular e efetivo para solucionar esse problema, \textit{client-side prediction} \cite{client-side-prediction}, baseia-se em prever os próximos \textit{inputs} imediatos dos jogadores e agindo antes mesmo que os dados de seu oponente sejam transmitidos; desta forma, removendo a necessidade inicial de espera. Uma vez que os \textit{inputs} são recebidos, estes são comparados com a previsão realizada e, caso sejam incongruentes entre si, o jogo é retornado ao estado anterior do momento da previsão inicial.

Por possuir requisitos semelhantes de baixa latência, a mesma implementação baseada em previsões tem o potencial de resolver o problema descrito anteriormente para ambientes musicais colaborativos \textit{online}. Caso seja possível prever os próximos sinais digitais produzidos pelos artistas remotos, não haveria necessidade de espera e, portanto, a latência de rede tornaria-se irrelevante.

Evidentemente, as diferenças entre os contextos de \textit{videogames} e práticas musicais não são negligenciáveis. Ao contrário dos jogos, por apresentar uma linearidade no tempo, não é possível retornar ao último momento da música anterior à previsão imediata sem atrapalhar a performance dos artistas. Além disso, a natureza discreta dos \textit{inputs} dos \textit{videogames} e a quantidade bruta de dados produzida é ínfima em comparação à áudios digitais - estima-se que os jogadores profissionais de \textit{Super Smash Bros. Melee} mais técnicos produzem em média 6 \textit{inputs} por segundo \cite{melee_inputs_per_second}; uma transmissão de áudio com \textit{sample rate} de 44,1 Khz produz, por definição, 44.100 diferentes valores no mesmo espaço de tempo \cite{jukebox_dimension}. Portanto, a aplicação do algoritmo de \textit{client-side prediction} para transmissão de música \textit{online} não é trivial, e a necessidade que o modelo preditivo seja o mais acurado possível é ainda mais importante.

Dois ciclos de estudos foram realizados para explorar essa abordagem. No primeiro ciclo, foram utilizados métodos de aprendizagem de máquina da biblioteca Keras \cite{keras}, especificamente LSTM (\textit{Long Short-Term Memory})\cite{lstm}. Modelos foram construídos e treinados com cortes de arquivos de áudio no formato WAV. A partir desses modelos, novas previsões de sequência foram geradas.

O segundo ciclo seguiu uma abordagem de criar um banco de dados de referência e, para cada nova entrada, identificou-se o corte mais semelhante. A semelhança foi calculada a partir um algoritmo de identificação baseado em DTW (\textit{Dynamic Time Warping})\cite{dtw}, implementado pela biblioteca Librosa \cite{librosa}. Possuindo uma janela de referência, a próxima sequência do banco de dados foi assim escolhida como predição da continuidade da música.

Para avaliar os dois modelos, utilizamos dois critérios de sucesso: (1) quão correta foram as previsões realizas e; (2) o tempo necessário para gerar as previsões. Dados esses parâmetros, o modelo gerador com LSTM não teve boa performance, falhando nos dois critérios. Entretanto, o modelo indexador com DTW mostrou-se bastante promissor, tendo sucesso nas duas medidas. Em particular para músicos que performam a trilha de base das músicas, com menos improvisos, esse método pode ser explorado futuramente em ambientes reais.

No \chapref{chap:context}, contextualizamos o problema enfrentado pelos ambientes musicais colaborativos \textit{online} atualmente, explorando as atuais abordagens utilizadas que procuram solucioná-lo. No \chapref{chap:solution_propositon}, detalhamos nossa solução proposta - uma adaptação da técnica \textit{client-side prediction} para os ambientes musicais - entrando também em detalhes sobre os dois modelos preditivos experimentados - o modelo gerador de novas sequências (LSTM) e o identificador e indexador de sequências (DTW). No \chapref{chap:lstm} e \chapref{chap:dtw}, exploramos o primeiro e segundo ciclo de estudos respectivamente, onde simulamos um ambiente virtual para testar os modelos preditivos propostos. Finalmente, no \chapref{chap:conclusion}, realizamos uma análise crítica dos resultados apresentados em nossos experimentos, apontando, também, possíveis trabalhos futuros para aprimorá-los.