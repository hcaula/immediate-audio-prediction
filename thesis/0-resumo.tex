\resumo
\addcontentsline{toc}{chapter}{Resumo}
Devido ao baixo tempo humano de reação à estímulos auditivos, para performar em conjunto com outros artistas, músicos necessitam que haja pouca latência entre a saída de dos instrumentos de seus colegas e seu retorno. Dessa forma, proporcionar um ambiente colaborativo em tempo real via Internet torna-se um desafio pertinente na área de Computação Musical e Rede de Computadores. Algumas abordagens procuram otimizar a conexão entre os computadores construindo infraestruturas dedicadas ou abandonam o requisito de tempo real ao entregar experiências assíncronas. No entanto, tais soluções não abrangem, de forma síncrona, casos em que não haja acesso a uma conexão de alta velocidade ou que exista uma grande distância entre os participantes.

Este trabalho, portanto, propõe a possiblidade de predizer as próximas sequências imediatas de áudio, baseando-se em entradas passadas, utilizando métodos estatísticos e aprendizagem de máquina. De tal forma, não sendo necessário a espera pela saída do cliente transmissor, reduz-se a percepção de latência do ouvinte.
\begin{keywords}
latência, áudio, predição de sequência, streaming, computação musical
\end{keywords}
