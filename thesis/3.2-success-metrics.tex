\section{Métricas de sucesso dos modelos preditivos}
\label{sec:success_metrics}

Como mencionado, existem alguns requisitos que os modelos preditivos propostos precisam cumprir para serem considerados bem sucedidos. Em cada ciclo, diferentes metodologias foram utilizadas para demonstrar essas métricas.

\subsection{Corretude das previsões}
\label{subsec:prediction_correctness}

Devido à característica de linearidade de tempo que a música possui, não é possível ``retornar'' a um estado passado da música. Dessa forma, é importante que os modelos sejam os mais acurados possíveis.

Dada a natureza da predição, não é esperado que as sequências geradas sejam integralmente fidedignas às sequências reais. No entanto, isso não é um requisito para produzir resultados positivos, sendo realmente importante que os áudios previstos tenham a mesma ``intenção'' que a reprodução original.

Dessa forma, o critério de sucesso dessa métrica é, para as sequências previstas, que seja imperceptível, em tempo real, que haja diferenças entre as sequências originais; de forma que um músico, ao ouvir ambas as sequências, reagiria musicalmente da mesma maneira ou de maneira similar.

Note que, por se tratar de um critério abstrato, quantificar essa métrica pode não representar bem o seu sucesso. Em cada um dos ciclos, utilizamos diferentes metodologias (demonstrados em \secref{sec:lstm_metodology} e \secref{sec:dtw_metodology}) para medi-la, visando sempre satisfazer o critério estabelecido.

\subsection{Tempo de geração de previsões}
\label{subsec:time_metric}

Supondo que \textit{t} é o tempo escolhido das janelas de previsão de áudio e \textit{p} é o tempo levado para gerar essas previsões, é imprescindível para o sucesso dos modelos preditivos que $t \leq p $. Caso contrário, o tempo excedente de processamento cria um \textit{delay} entre o que está sendo executado pelo músico remoto e o que está sendo reproduzido localmente e, consequentemente, causando dessincronia entre os músicos.

Ao contrário do critério de corretude na \secref{subsec:prediction_correctness}, essa métrica consegue ser medida precisamente. Será considerado bem sucedida o modelo preditivo que produzir sequências em um tempo menor que a duração do áudio gerado.
