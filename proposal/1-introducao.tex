\chapter{Introdução}

A performance artística musical, quando praticada em conjunto, requer alto nível de colaboração entre os participantes. Em música, sobretudo gêneros com tendências improvisacionais como \textit{jazz}, \textit{blues} e \textit{rock}, o ato de ouvir e reagir ao som de seus companheiros é tão importante quanto aquele produzido individualmente. Dessa forma, o \textit{feedback} auditivo de baixa latência dos instrumentos tocados é fundamental para que haja uma sensação fluida entre os participantes.

Normalmente, músicos performando em conjunto em um mesmo ambiente físico raramente experienciarão problemas relacionados a latência. No entanto, em um contexto de distanciamento social, encorajado durante à Pandemia de COVID-19 no ano de 2020, músicos ao redor do mundo viram-se obrigados a transferirem esse ambiente para um virtual \textit{online}.

% Devido à abordagem de "melhor esforço" em que a Internet foi originalmente projetada - sob a suposição de que não é possível garantir o recebimento de todos os pacotes transmitidos - protocolos de transmissão de voz como \textit{VoIP} (\textit{Voice over Internet Protocol}) e serviços provedores videoconferências necessitam implementar medidas compensatórias, como grandes \textit{buffers} de rede e retransmissão de pacotes. Tais medidas, consequentemente, garantem a corretude dos dados transmitidos, sob o custo do aumento na latência total \cite{carot_low_latency}. Para aplicações comuns, essa troca é aceitável; entretanto, no contexto da arte musical colaborativa, mostra-se inviável.

Aplicações comerciais de videoconferências, como o \textit{Zoom}, \textit{Google Meet} e \textit{FaceTime}  possuem sensibilidade de tempo para manter conversas compreensíveis - a \textit{Cisco} define a latência máxima aceitável de uma implementação \textit{VoIP} em até 150 ms \cite{cisco}. Este limite, apesar de relativamente baixo, pode ser alcançado por conexões de velocidades medianas, implementações de processamento de áudio e infraestruturas de rede compartilhadas. No entanto, para a prática colaborativa musical, onde tolerância máxima é bastante restrita - variando entre 10 ms e 55 ms \cite{mcphearson} - mostra-se inviável. Em ambientes de alta latência, músicos tendem a perceber incômodos e mudam a forma sobre como performam para adptarem-se. \cite{carot_low_latency}.

Para lidar com estes problemas, \textit{softwares} voltados especificamente para a colaboração musical \textit{online} apresentam uma variedade de abordagens diferentes. \textit{LoLa} \cite{lola}, \textit{SoundJack} \cite{soundjack} e \textit{JamKazam} \cite{jamkazam}, por exemplo, implementam otimizações na camada de rede - como conectar clientes diretamente entre si via \textit{P2P} (\textit{peer-to-peer}) - oferecendo latências aceitáveis entre distâncias medianas. Outras aplicações, como o \textit{Jammr} \cite{jammr}, dispensam o requisito de tempo real e apresentam soluções assíncronas, onde os músicos ouvem os últimos quatro compassos tocados por seus companheiros em um \textit{loop} contínuo.

No entanto, tais abordagens não abrangem casos onde não é possível ter acesso a conexões dedicadas ou os músicos residem entre grandes distâncias, ainda mantendo uma performance síncrona.

Ao observar o contexto de videogames, encontramos requisitos de latência similares. Gêneros que utilizam reações como mecânica de jogabilidade, como luta e FPS (\textit{first-person shooter}), para oferecerem aos jogadores uma experiência fluida, necessitam de latências máximas de até 100 ms \cite{pubnub}. O algoritmo mais popular e efetivo para solucionar esse problema, \textit{Rollback Netcode} \cite{rollback}, baseia-se em prever os próximos \textit{inputs} imediatos dos jogadores e agindo antes mesmo que os dados de seu oponente sejam transmitidos; desta forma, removendo qualquer necessidade de espera. Uma vez que os \textit{inputs} são recebidos, estes são comparados com a previsão realizada e, caso sejam incongruentes entre si, o jogo é retornado ao estado anterior do momento da previsão inicial.

Por possuir contextos semelhantes, a mesma implementação baseada em previsões tem o potencial de resolver o problema descrito anteriormente para ambientes musicais colaborativos \textit{online}. Caso seja possível prever os próximos sinais digitais produzidos pelos artistas remotos, não haveria necessidade de espera e, portanto, a latência de rede tornaria-se irrelevante. É  evidente que, entretanto, por apresentar uma linearidade no tempo, não é possível retornar ao último momento da música anterior à previsão imediata. Portanto, é necessário que o modelo preditivo seja o mais acurado possível, visando minimizar a quantidade total de erros.
