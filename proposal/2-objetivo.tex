\chapter{Objetivo}

O desafio de redução de latência em \textit{streamings} de áudio já possui o interesse da comunidade musical e, em um contexto de distanciamento social, torna-se ainda mais relevante. Dessa forma, estudos que propõem novas abordagens para alcançar esse objetivo, possibilitando a criação ambientes virtuais musicais viáveis e acessíveis, possuem sua devida importância.

Este trabalho de graduação propõe, portanto, validar a viabilidade de um algoritmo inspirado no \textit{Rollback Netcode} aplicado no contexto de performances musicais, gerando previsões de sequências imediatas de sinais digitais de áudio.

Pretende-se pesquisar e aplicar métodos no estado da arte em previsão de sequência, sejam supervisionadas ou não, em diferentes arquivos de áudio contendo gravações de instrumentos musicais. Avaliará-se, primordialmente, a taxa de acerto das previsões. Diferentes contextos serão estudados, simulando diferentes latências, janelas de tempo de previsão, qualidade do arquivo de áudio de entrada e do \textit{ouput} gerado, assim como a eficácia entre diferentes algoritmos de previsão.

Por último, concluirá-se se, de acordo com as taxas de acerto, uma aplicação que utilize tal algoritmo pode ser viável para a performance musical em conjunto em um ambiente \textit{online}. É importante mencionar, no entanto, que este trabalho não propõe a implementação da aplicação em si, apenas da validação da hipótese proposta.
